\documentclass[11pt,a4paper]{ivoa}
\input tthdefs

\title{TableReg: Registering TAP-Queriable Tables Conforming to Standard
Schemas}

% see ivoatexDoc for what group names to use here; use \ivoagroup[IG] for
% interest groups.
\ivoagroup{Registry}

\author[http://www.ivoa.net/cgi-bin/twiki/bin/view/IVOA/MarkusDemleitner]{Markus Demleitner}

\editor{Markus Demleitner}

\previousversion{This is the first public release}


\begin{document}
\begin{abstract}
An increasingly popular pattern in the Virtual Observatory, pioneered
by Obscore, is to define a schema for one or more tables in a database
and then publish data by making it accessible in tables conforming to
that schema in TAP services.  This document discusses how such resources
should be represented in the VO Registry to facilitate data discovery,
in particular global, all-VO data discovery.

It turns out that the existing registration patterns for Obscore,
RegTAP, and EPN-TAP require some adjustments.  The document therefore
also proposes transition strategies for these.
\end{abstract}


\section*{Conformance-related definitions}

The words ``MUST'', ``SHALL'', ``SHOULD'', ``MAY'', ``RECOMMENDED'', and
``OPTIONAL'' (in upper or lower case) used in this document are to be
interpreted as described in IETF standard RFC2119 \citep{std:RFC2119}.

The \emph{Virtual Observatory (VO)} is a
general term for a collection of federated resources that can be used
to conduct astronomical research, education, and outreach.
The \href{https://www.ivoa.net}{International
Virtual Observatory Alliance (IVOA)} is a global
collaboration of separately funded projects to develop standards and
infrastructure that enable VO applications.


\section{Introduction}

Beginning with Obscore 1.0 \citep{2011ivoa.spec.1028T}, an increasing
number of Virtual Observatory standards at their core just define a
table schema -- understood here as a well-defined set of columns within
one or more relations -- and rely on TAP \citep{2019ivoa.spec.0927D} to
let clients actually run queries.  Standards of this type include:

\begin{itemize}
\item Obscore \citep{2017ivoa.spec.0509L} -- a table for metadata of
observational data products
\item RegTAP \citep{2019ivoa.spec.1011D} -- a 13-table schema with
metadata of VO resources
\item ObsLocTAP \citep{2021ivoa.spec.0724S} -- a table schema to
communicate plans for observations and metadata of completed
observations
\item EPN-TAP \citep{2022ivoa.spec.0822E} -- a table schema for solar
system data
\end{itemize}

More such standards are currently being developed, such as LineTAP
\citep{wd:linetap23} and the Obscore extension for radio data.

Of course, resources complying to these standards must be made
discoverable to be useful.  Both Obscore and RegTAP have employed the
\xmlel{dataModel} element specifically introduced into TAPRegExt
\citep{2012ivoa.spec.0827D} to declare the presence of tables adhering
to a standard schema in a TAP service.

Both Obscore and RegTAP are singletons with fixed names, i.e., once one
has discovered a TAP service ``supporting'' a data model, it is clear
how to query it.  However, even for these, the \xmlel{dataModel}-scheme
has severe shortcomings:

\begin{enumerate}
\item Lack of resource metadata: In resource records located during
discovery, the global VOResource metadata (title, authors, and perhaps
most importantly coverage in space, time and spectrum) are these of the
TAP service, which the tables may share with any number of other tables.

\item Unclear relationships: In particular for Obscore, a severe
shortcoming is that data collection records can only generically say
that they are served by the TAP service (cf.~\citet{2019ivoa.spec.0520D}
for the general scheme of relating data collections and services).  From
that, clients cannot deduce whether the data is available through, say,
obscore, or only in some custom table.

\item Logic: Adherence to a data model simply is not a property of a
TAP service.  It is a property of a specific table or schema.
\end{enumerate}

EPN-TAP then employed the table-in-TAP scheme without the constraint to
a singleton; since each data collection has an EPN-TAP table of its own,
the discovery process somehow also had to yield a table name.  As a
solution, the standard forsees using the table's \xmlel{utype} field in
the VODataService \citep{2010ivoa.spec.1202P} \xmlel{tableset}.

This mode of discovery was subsequently also employed in ObsLocTAP and
LineTAP.  It still has a drawback, though: Clients discovering a service
with a \xmlel{table} (or \xmlel{schema} in the case of RegTAP) with the
bespoke \xmlel{utype} have a hard time determining whether what they
have found is the data collection's record (and hence the global
metadata pertains to the data collection itself) or the record of the
TAP service serving the data collection (in which case the global
metadata pertains to the TAP service and will be essentially unrelated
to the data collection in question).

This note remedies that by prescribing that for resource records for
standard-compliant tables, the resource type \xmlel{vs:CatalogResource}
must be used.

In Sect.~\ref{sect:norms}, we review the proposed scheme and give
boilerplate text to include in standards following it.  One reason to
introduce the scheme is to enable expressing inclusion relationships
between resources for the benefit of clients doing global discovery.
Sect.~\ref{sect:rels} discusses this mechanism in more detail.  Finally,
Sect.~\ref{sect:transition} addresses the question of how to transition
from what the standards currently say (and the services and clients
implement) to a VO adopting the scheme proposed here.

\section{Registering and Discovering Standard Tables}
\label{sect:norms}

\section{Expressing Relationships Between Tables}
\label{sect:rels}

\section{Transitioning to a TableReg World}
\label{sect:transition}


\appendix
\section{Changes from Previous Versions}

No previous versions yet.
% these would be subsections "Changes from v. WD-..."
% Use itemize environments.


% NOTE: IVOA recommendations must be cited from docrepo rather than ivoabib
% (REC entries there are for legacy documents only)
\bibliography{ivoatex/ivoabib,ivoatex/docrepo,local}


\end{document}
